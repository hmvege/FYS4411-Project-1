\documentclass[11pt]{article}

\usepackage[utf8]{inputenc}
\usepackage{amsmath}
\usepackage{mathtools}
\usepackage{amsfonts}
\usepackage{graphicx}
\usepackage{cite}
\usepackage{url}
\usepackage{color}
\usepackage{float}
\usepackage{hyperref}

\overfullrule=2cm

\newcommand{\husk}[1]{\color{red} #1 \color{black}}
\newcommand{\expect}[1]{\langle{#1}\rangle}

\title{Project 1 in FYS4411: Computational Physics 2}
\author{}

\date{\today}
\begin{document}
\maketitle

\begin{abstract}
Nan
\end{abstract}

\section{Introduction}
For this project our main task was to explore interacting systems of electrons in two dimensions, quantum dots. Such systems have a wide range of applications, as they can  For exploring such systems, we were to employ the Hartree-Fock method. 

\section{Theory}
\subsection{Quantum dots}
\subsection{Quantum Harmonic Oscillator}
\subsection{Hartree-Fock}
\subsubsection{Hartree-Fock algorithm}

\section{Results}
\subsection{Unperturbed results}

\section{Conclusions and discussions}

\end{document}