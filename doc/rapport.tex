\documentclass[11pt,twocolumn]{article}

\usepackage[utf8]{inputenc}
\usepackage{amsmath}
\usepackage{mathtools}
\usepackage{amsfonts}
\usepackage{graphicx}
\usepackage{cite}
\usepackage{url}
\usepackage{color}
\usepackage{float}
\usepackage{hyperref}

\overfullrule=2cm

\newcommand{\husk}[1]{\color{red} #1 \color{black}}
\newcommand{\expect}[1]{\langle{#1}\rangle}

\title{Project 1 in FYS4411: Computational Physics 2}
\author{}

\date{\today}
\begin{document}
\maketitle

\begin{abstract}
NAN
\end{abstract}

\section{Introduction}
For this project our main task was to explore interacting systems of electrons in two dimensions, quantum dots. Such systems have a wide range of applications, as they can  For exploring such systems, we were to employ the Hartree-Fock method. For our project, we have looked at electrons confined in a harmonic oscillator potential where every shell up to a chosen limit has been filled. By doing this, we have that the number of electrons confine to \textit{magic numbers}, $N = 2, 6, 12, 20, 30$ and so on.

In order to present my results, I will begin repeating the physics involved in this project, and finally go through the Hartree-Fock algorithm.

\section{Theory}
The full Hamiltonian for our quantum dot system is on the form
\begin{align}
	H &= H_0 + H_I \nonumber \\
	&= \sum^N_{i=0} \left( -\frac{1}{2}\nabla^2_i + \frac{1}{2}\omega^2 r^2_i \right) + \sum^N_{i<j}\frac{1}{r_{ij}},
	\label{eq:full-hamiltonian}
\end{align}
with natural units $\hbar = c = e = m_e = 1$. The first part $H_0$ is the unperturbed Hamiltonian, consisting of the kinetic energy and the harmonic oscillator potential. The $r_{ij}$ is defined as $r_{ij} = \sqrt{\mathbf{r}_1 - \mathbf{r}_2}$, while the $r_i$ is defined as $r_i = \sqrt{r^2_{i_x} + r^2_{i_y}}$. The harmonic oscillator potential has a oscillator frequency $\omega$. The sums run over all particles $N$.

An unperturbed two dimensional harmonic oscillator have energies given as
\begin{align}
	\varepsilon_{n_x,n_y} = \omega(n_x + n_y + 1)
	\label{eq:ho-energy-cartesian}
\end{align}

The wave function solution for a harmonic oscillator is given by the Hermite polynomials,
\begin{align}
	\phi_{n_x,n_y}(x,y) = AH_{n_x}(\sqrt{\omega}x)H_{n_y}(\sqrt{\omega}y)\times \nonumber \\
	\exp(-\omega(x^2 + y^2)/2)
\end{align}

As is evident in our calculations later on, we will be using polar coordinates to describe our system. Doing this changes our quantum numbers from $n_x$ and $n_y$ to $n$ and $m$, and the energies is now given by
\begin{align}
	\varepsilon_{n, m} = \omega(2n + |m| + 1)
	\label{eq:ho-energy-polar}
\end{align}

\subsection{Hartree-Fock}
In order to derive the Hartree-Fock equations, we begin by setting up the functional for the ground state energy which we are to minimize,
\begin{align}
	E_0 \leq E[\Phi] = \int \Phi^* \hat{H} \Phi d\tau.
	\label{eq:variational-principle}
\end{align}
with $d\tau = d\mathbf{r}_1d\mathbf{r}_2\dots d\mathbf{r}_N$ and $\Phi$ as a wave function function we wish to use to minimize the energy for. Inserting for the Hamiltonian we set up earlier \eqref{eq:full-hamiltonian}, we get
\begin{align}
	E[\Phi] = \int \Phi^* H_0 \Phi d\tau + \int \Phi^* H_I \Phi d\tau
	\label{eq:energy-functional}
\end{align}
We start by looking at the first term.





\husk{Considering} a basis transformation of $\phi_\lambda(x)$ to $\psi_p(x)$ through
\begin{align}
	\psi_p (x) = \sum_\lambda C_{p\lambda} \phi_\lambda(x),
	\label{eq:wavefunc-transform}
\end{align}
where the $C_{p\lambda}$ is an unitary matrix.

\subsubsection{Hartree-Fock algorithm}


\section{Results}
\subsection{Unperturbed results}
\subsection{Unit tests}

\section{Conclusions and discussions}

\section{Appendix A: mathematical formulas}
\subsection{Slater determinants}
A Slater determinant is an expression describing a multi-fermionic system in accordance to the Pauli principle. For system of $N$ particles, we have 
\begin{align}
	\begin{split}
		\Phi(x_1,x_2,\dots,x_N; \alpha,\dots,\sigma) =\\
		\frac{1}{\sqrt{N!}}
		\begin{vmatrix}
			\psi_\alpha(x_1) & \dots & \psi_\alpha(x_N) \\
			\dots & \dots & \dots \\
			\psi_\sigma(x_1) & \dots & \psi_\sigma(x_N) \\
		\end{vmatrix}
	\end{split}
	\label{eq:slater-determinant}
\end{align}
This can be written in a more compact form using the anti-symmetrization operator $\hat{A}$,
\begin{align}
	\hat{A} = \frac{1}{N!}\sum_P (-1)^P \hat{P}
	\label{eq:anti-symmetrization-operator}
\end{align}
Combining the Slater determinant \eqref{eq:slater-determinant} with $\hat{A}$, we get
\begin{align}
	&\Phi(x_1,\dots,x_N;\alpha,\dots,\nu) \nonumber \\
	&= \frac{1}{\sqrt{N!}}\sum_P (-1)^P \hat{P}\psi_\alpha(x_1)\dots\psi_\nu(x_N) \nonumber \\
	&= \frac{1}{\sqrt{N!}}\sum_P (-1)^P \hat{P} \prod_s \psi_s(x_s) \nonumber \\
	&= \sqrt{N!}\hat{A}\Phi_H,
	\label{eq:slater-determinant-compact}
\end{align}
where $\Phi_H$ is the wave function that is being permuted according to its components.

The $\psi$ is given by an orthogonal basis $\phi$
\begin{align}
	\psi_p = \sum_\lambda C_{p\lambda} \phi_\lambda
	\label{eq:basis-orthogonality}
\end{align}
where the $\lambda$ runs over all the single particle states $\alpha, \beta, \dots, \nu$. We have that $\phi_\lambda$ is an orthogonal basis, and we can show from this that $\psi_p$ also must be an orthogonal basis.
\begin{align*}
	\langle \psi_p | \psi_q \rangle &= \int \left( \sum_\lambda C^*_{p\lambda}\phi_\lambda \right) \left( \sum_\eta C_{q\eta}\phi_\eta \right) d\mathbf{r}_i \\
	&= \sum_{\lambda,\eta} C^*_{p\lambda} C_{q\eta} \int \phi_\lambda \phi_\eta d\mathbf{r}_i \\
	&= \sum_{\lambda,\eta} C^*_{p\lambda} C_{q\eta} \delta_{\lambda\eta}
\end{align*}



\end{document}